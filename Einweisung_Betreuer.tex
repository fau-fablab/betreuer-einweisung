%%%%%%%%%%%%%%%%%%%%%%%%%%%%%%%%%%%%%%%%%%%%%%%%
% COPYRIGHT: (C) 2012-2015 FAU FabLab and others
% Bearbeitungen ab 2015-02-20 fallen unter CC-BY-SA 3.0
% Sobald alle Mitautoren zugestimmt haben, steht die komplette Datei unter CC-BY-SA 3.0. Bis dahin ist der Lizenzstatus aller alten Bestandteile ungeklärt.
%%%%%%%%%%%%%%%%%%%%%%%%%%%%%%%%%%%%%%%%%%%%%%%%


\newcommand{\basedir}{fablab-document}
\documentclass{\basedir/fablab-document}

\usepackage{amssymb} % Symbole für Knöpfe
\usepackage{subfigure,caption}
\usepackage{eurosym}
\usepackage{tabularx} % Tabellen mit bestimmtem Breitenverhältnis der Spalten
\usepackage{wrapfig} % Textumlauf um Bilder
\renewcommand{\texteuro}{\euro}
\newcommand{\todoRot}[1]{\textbf{\color{red}{TODO: #1}}}
%%%%% HACKY
\usepackage{todonotes}
\let\todox\todo
\renewcommand\todo[1]{\todox[inline]{#1}}
%%%%%
\newcommand{\pfeil}{\ensuremath{\rightarrow}}
\newcommand{\kommando}[1]{\texttt{#1}}

% \linespread{1.2}

\fancyhead[C]{\todoRot{unfertig!}}
\date{2014}
\author{kontakt@fablab.fau.de}
\title{Betreuen im FabLab}

\begin{document}
~
\section{asdf}
\subsection{asdf}
\todo{Präsentation anlegen}
\todo{Website bedienen}
\todo{svn und der ganze Krams}

\subsection{Betrieb}
Wenn zuwenig Geld dabei, Zettel in die Kasse legen mit Name, Kontaktdaten, Betrag. Im Terminal nur das eingeben, was auch wirklich in die Kasse gelegt wurde!

Dinge ausleihen
\begin{itemize}
 \item In Ausleihliste im Kassenbuch-Ordner eintragen
 \item ausreichend Pfand verlangen (es sei denn, der Betreuer möchte selbst dafür haften, z.B. weil er den Ausleiher persönlich kennt)
 \item Weitere Regeln siehe Ausleihliste. Es gibt kein Recht auf Ausleihen, der Lab-Betrieb darf dadurch nicht behindert werden.
\end{itemize}
 , 

Ordnung halten. Wenn andere Leute Unordnung machen, diese höflich aber bestimmt zum Aufräumen auffordern.

Darauf achten, dass keine Nutzer ohne Einweisung alleine an Geräten arbeiten. Im Zweifel lieber einmal zuviel nachfragen, ob der Nutzer die Einweisung schon unterschrieben hat. (gelbe/rote Hinweisschilder beachten)

Müll leeren, sobald er recht voll ist. Dann auch gleich alle Mülleimer im Lab und Besprechungsraum mit einsammeln. Im Zweifelsfall hat der Betreuer am Anfang des Lab-Termins den Müll zu leeren.

Der Besprechungsraum gehört nicht dem Lab, wir sind nur \enquote{geduldet}. Deshalb ist dort besonders auf Sauberkeit zu achten -- ganz egal woher die Unordnung kommt, saubermachen!
Raumreservierungen für den Besprechungsraum sind über Max möglich. Da andere Leute Zugang zum Besprechungsraum haben, muss die Zwischentür beim Verlassen des Labs geschlossen werden.

\subsection{Zugang}
Im Kalender eingetragene Termine haben Vorrang. Achtung, \enquote{Schließungszeiten} erscheinen nicht im Kasten auf der Startseite, sondern nur unter Termine.

Es muss immer eine Person mit Schließberechtigung im Lab sein (Es sei denn, der Betreuer muss mal kurz aufs Klo). Wer Leute reinlässt, muss sich um sie kümmern -- mindestens soweit, dass hinterher keine Unordnung ist. Wer geht, muss sicherstellen dass das Lab leer ist oder die Aufsicht an jemanden mit Schließberechtigung übergeben.

Beim Gehen alles ausschalten (siehe Zettel an der Tür).

\subsection{Blindtext}
\blindtext
\end{document}
