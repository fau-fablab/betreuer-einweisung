%%%%%%%%%%%%%%%%%%%%%%%%%%%%%%%%%%%%%%%%%%%%%%%%
% COPYRIGHT: (C) 2012-2015 FAU FabLab and others
% Bearbeitungen ab 2015-02-20 fallen unter CC-BY-SA 3.0
% Sobald alle Mitautoren zugestimmt haben, steht die komplette Datei unter CC-BY-SA 3.0. Bis dahin ist der Lizenzstatus aller alten Bestandteile ungeklärt.
%%%%%%%%%%%%%%%%%%%%%%%%%%%%%%%%%%%%%%%%%%%%%%%%


\newcommand{\basedir}{fablab-document}
\documentclass{\basedir/fablab-document}

\usepackage{amssymb} % Symbole für Knöpfe
\usepackage{subfigure,caption}
\usepackage{eurosym}
\usepackage{tabularx} % Tabellen mit bestimmtem Breitenverhältnis der Spalten
\usepackage{wrapfig} % Textumlauf um Bilder
\renewcommand{\texteuro}{\euro}
\newcommand{\todoRot}[1]{\textbf{\color{red}{TODO: #1}}}
%%%%% HACKY
\usepackage{todonotes}
\let\todox\todo
\renewcommand\todo[1]{\todox[inline]{#1}}
%%%%%
\newcommand{\pfeil}{\ensuremath{\rightarrow}}
\newcommand{\kommando}[1]{\texttt{#1}}

% \linespread{1.2}

\fancyhead[C]{\todoRot{unfertig!}}
\date{2014}
\author{kontakt@fablab.fau.de}
\title{Betreuen im FabLab}

\begin{document}
~
\section{Vernetzung}
\subsection{Treffen}
Das FabLab hat zwei Arten von Treffen: das \textbf{Betreuertreffen} und das \textbf{Orgatreffen}.
Am Betreuertreffen treffen sich alle Mitglieder des FabLabs um über aktuelle Gegebenheiten informiert zu werden,
grundsätzliche Entscheidungen zu treffen und das Lab aufzuräumen oder zu erweitern.
Das Orgatreffen dient dazu, um über tagtägliche Gegebenheiten zu entscheiden und langfristige Konzepte auszuarbeiten.

\subsection{Mailingliste und Telegram}
Die Mailingliste ist das Hauptkommunikationsmittel.
Abonnieren ist unter \url{https://lists.fau.de/cgi-bin/listinfo/fablab-aktive} möglich.
Achtung, über die Mailingliste kommen relativ viele Mails, es empfiehlt sich die Mails nach \texttt{list-id} zu filtern.

Für kurzfristige Kommunikation gibt es eine Telegram-Gruppe.
Um diese zu abonnieren, einfach auf der Mailingliste fragen.
Wichtiges sollte dennoch auf die Mailingliste.
\subsection{Website}
Wir haben eine Website um über aktuelle Neuerungen zu berichten und die Termine anzuzeigen.
Du findest sie unter \url{https://fablab.fau.de}
Bitte lege dir einen Benutzer auf der Website an und ergänze die Daten um ein Bild, deine E-Mail-Adresse und deine Telefonnummer.
Diese Informationen werden nur intern angezeigt und dienen dazu, um dich schnell erreichen zu können.

Außerdem kannst du mit einem Benutzer der als Betreuer markiert ist Beiträge auf der Website anlegen.

\subsection{GitHub}
Viele unserer Dokumente und selbst entwickelten Programme liegen auf GitHub unter \url{https://github.com/fau-fablab/}.
Wenn du dir einen GitHub-Account anlegst, kannst du Zugriff erhalten.

\subsection{Server}
Um \enquote{von außen} auf die Dateifreigabe zuzugreifen gibt es einen shell-Server.
Auf diesen kannst du dich per SSH einloggen und Programmen, die das Protokoll SFTP unterstützen Daten übertragen.
Um dich auf dem Server zu registrieren, kontaktiere bitte \url{admins@fablab.fau.de}.

\section{Lab}
\subsection{Betrieb}
Wenn zuwenig Geld dabei, Zettel in die Kasse (hinten in der Betreuerschublade) legen mit Name, Kontaktdaten, Betrag.
Im Terminal nur das eingeben, was auch wirklich in die Kasse gelegt wurde!

Dinge ausleihen
\begin{itemize}
 \item In Ausleihliste im Kassenbuch-Ordner eintragen
 \item ausreichend Pfand verlangen (es sei denn, der Betreuer möchte selbst dafür haften, z.B. weil er den Ausleiher persönlich kennt)
 \item Weitere Regeln siehe Ausleihliste. Es gibt kein Recht auf Ausleihen, der Lab-Betrieb darf dadurch nicht behindert werden.
\end{itemize}

Ordnung halten. Wenn andere Leute Unordnung machen, diese höflich aber bestimmt zum Aufräumen auffordern.

Darauf achten, dass keine Nutzer ohne Einweisung alleine an Geräten arbeiten. Im Zweifel lieber einmal zuviel nachfragen, ob der Nutzer die Einweisung schon unterschrieben hat. (gelbe/rote Hinweisschilder beachten)

Müll leeren, sobald er recht voll ist. Dann auch gleich alle Mülleimer im Lab und Besprechungsraum mit einsammeln. Im Zweifelsfall hat der Betreuer am Anfang des Lab-Termins den Müll zu leeren.

Essensecke. Die Essensecke ist immer sauber zu halten. Ganz besonders wichtig ist darauf zu achten, dass Chemie- und Essensecke sich nicht \enquote{vermischen}.

Der Besprechungsraum gehört uns gemeinsam mit der Studierendenvertretung. Deshalb ist dort besonders auf Sauberkeit zu achten -- ganz egal woher die Unordnung kommt, saubermachen!
Raumreservierungen für den Besprechungsraum sind über Max möglich. Da andere Leute Zugang zum Besprechungsraum haben, muss die Zwischentür beim Verlassen des Labs geschlossen werden.

\subsection{Zugang}
Im Kalender eingetragene Termine haben Vorrang. Achtung, \enquote{Schließungszeiten} erscheinen nicht im Kasten auf der Startseite, sondern nur unter Termine.

Es muss immer eine Person mit Schließberechtigung im Lab sein (Es sei denn, der Betreuer muss mal kurz aufs Klo). Wer Leute reinlässt, muss sich um sie kümmern -- mindestens soweit, dass hinterher keine Unordnung ist. Wer geht, muss sicherstellen dass das Lab leer ist oder die Aufsicht an jemanden mit Schließberechtigung übergeben.

Beim Gehen alles ausschalten (siehe Zettel an der Tür).


\ccLicense{betreuer-einweisung}{Einweisung Betreuer}

Dieses Dokument stammt aus fau-fablab/betreuer-einweisung@\Revision{}.

\end{document}
